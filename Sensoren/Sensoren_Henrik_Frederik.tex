\documentclass[12pt,a4paper]{article}
 
\usepackage{float}
%für feststellen der figures und tables [H] dranschreiben
\usepackage{units}
%wird so benutzt: 
%\unit[value/Zahl]{dimension/Einheit} oder 
%\unitfrac[value/Zahl]{dimension/Einheit num/Zähler}{dimension/Einheit denum/Nenner} oder
%\nicefrac[fontcommand/Schriftart]{dimension/Einheit num/Zähler}{dimension/Einheit denum/Nenner}

\usepackage{caption}
\usepackage{subcaption}

\usepackage[left=2cm,right=2cm,top=2cm,bottom=2cm]{geometry}
\usepackage[utf8]{inputenc}
\usepackage[T1]{fontenc}
\usepackage{lmodern}
\usepackage[ngerman]{babel}
\usepackage{amsmath}
\usepackage{graphicx}
 
\title{Sensoren}
\author{Frederik Strothmann, Henrik Jürgens}
\date{\today}
%niemals zwei überschriften direkt übereinander schreiben, also immer mindestens in einem satz was sinnvolles unter jede überschrift schreiben (bei den versuchen z.B. das versuchsziel) 
\begin{document}
%deckblatt erstellen.
\maketitle
\newpage
\tableofcontents
\newpage
\section{Einleitung}
%einleitung zu dem experiment.
%auf die einstellungen, die vor dem versuch gemacht werden, eingehen oder auf eine anleitung dazu verweisen
%es soll immer erwähnt werden um was es in dem Versuch geht und wie das relisiert werden soll
%---------------------------------------------------------------------------------------------
%hinter der einleitung kann der allgemeine theoretische hintergrund in einer zusätzlichen section erklärt werden
%1-----------------------------------------------1

In diesem Versuch werden Verschieden Sensoren zur Messung von Licht, Temperatur, Induktion, Kraft, Luftdruck/Schall und Feuchte. Dazu werden jeweils unterschiedliche Sensoren getestet.


\section{Lichtmessung}


\subsection{Photowiderstand}
%kurz das ziel dieses versuchsteiles ansprechen, damit keine zwei überschriften direkt übereinander stehen!
%bei schwierigeren versuchen kann auch der theoretische hintergrund erläutert werden. (mit formeln, herleitungen und erklärungen)

Der Photowiderstand ist ein Halbleiterverbindung, der bei Lichteinstrahlung seinen Widerstand stark verändert. Cadmiumsulfid ist eine der Halbleiterverbindungen, aus denen ein Photowiderstand hergestellt werden kann.

\subsubsection{Verwendete Geräte}
%(immer) eine skizze oder ein foto einfügen, die geräte/materialien !nummerieren! und z.b. eine legende dazu schreiben, besser wäre es das ganze in einem Fließtext gut zu beschreiben.
%falls am anfang des versuches nicht klar ist, was alles verwendet wird, wenn möglich erst am ende ein großes foto von den verwendeten materialien machen!\\

Es wird ein Photowiderstand und ein DMM verwendet.

\subsubsection{Versuchsaufbau}
%skizze zum versuchsaufbau (oder foto) einfügen,   es muss erklärt werden wie das ganze funktioniert und welche speziellen einstellungen verwendet wurden (z.b. welche knöpfe an den geräten für die messung verdreht wurden)

1 ist der Photowiderstand und 2 ist das DMM.

\begin{figure}[H] 
  \centering
    \includegraphics[ scale = 0.7]{Auf_1_2.PNG}
  	\caption[Schaltskizze zur Messung des Photowiderstandes]{Schaltskizze zur Messung des Photowiderstandes}
  \label{fig:auf_1_2}
\end{figure}

\subsubsection{Versuchsdurchführung}
%erklären, !was! wir machen, !warum! wir das machen und mit welchem ziel
%(wichtig) präzize erklären, wie bei dem versuch vorgegangen und was gemacht wurde

Der Photowiderstand wird in das DMM eingesteckt und der Widerstand bei starker Beleuchtung (Fenster, Lampe), mittlerer Beleuchtung und schwacher Beleuchtung (abgedeckt mit der Hand). Dann wird noch ein Lichtstärkefilter von 20\%-100\%, der in 20\% Schritten skaliert ist verwendet und für jede Einstellung der Widerstand gemessen.

\subsubsection{Messergebnisse}
%die messwerte in !übersichtlichen! tabellen angegeben
%zu viele kleine tabellen in große tabellen überführen!
%zu große tabellen mit dem [scale]-befehl scalieren oder (falls zu lang) in zwei kleinere tabellen aufteilen
%(wichtig) vor !jeder! tabelle sagen, was gemessen wurde und wie die fehler gewählt wurden und ausreichend !erklären!, !warum! wir unsere fehler grade so gewählt haben
\subsubsection{Auswertung}
%zuerst !alle! errechneten werte entweder in ganzen sätzen aufzählen, oder in tabellen (übersichtlicher) dargestellen, sowie auf die verwendeten formeln verweisen (die referenzierung der formel kann in der überschrift stehen)
%kurz erwähnen (vor der tabelle), warum wir das ganze ausrechnen bzw. was wir dort ausrechnen
%danach histogramme und plots erstellen, wobei wenn möglich funktionen durch die plots gelegt werden (zur not können auch splines benutzt werden, was aber angegeben werden muss)
%bei fits immer die funktion und das reduzierte chiquadrat mit angegeben, wobei auf verständlichkeit beim entziffern der zehnerpotenzen geachtet werden muss z.b. f(x)=(wert+-fehler)\cdot10^{irgendeine zahl}\cdot x + (wert+-fehler)\cdot10^{irgendeine zahl}
%bei jedem fit erklären, nach welchem zusammenhang gefittet wurde und warum!
%bei plots darauf achten, dass die achsenbeschriftung (auch die tics) die richtige größe haben und die legende im plot nicht die messwerte verdeckt
%kurz die aufgabenstellung abhandeln
%2-----------------------------------------------2
\subsubsection{Diskussion}
%(immer) die gemessenen werte und die bestimmten werte über die messfehler mit literaturwerten oder untereinander vergleichen
%in welchem fehlerintervall des messwertes liegt der literaturwert oder der vergleichswert?
%wie ist der relative anteil des fehlers am messwert und damit die qualität unserer messung?
%in einem satz erklären, wie gut unser fehler und damit unsere messung ist
%kurz erläutern, wie systematische fehler unsere messung beeinflusst haben könnten
%(wichtig) zum schluss ansprechen, in wie weit die ergebnisse mit der theoretischen vorhersage übereinstimmen
%--------------------------------------------------------------------------------------------
%falls tabellen mit den messwerten zu lang werden, kann die section mit den messwerten auch hinter der diskussion angefügt bzw. eine section mit dem anhang eingefügt werden.
%1-----------------------------------------------1



\subsection{Photodiode}
%kurz das ziel dieses versuchsteiles ansprechen, damit keine zwei überschriften direkt übereinander stehen!
%bei schwierigeren versuchen kann auch der theoretische hintergrund erläutert werden. (mit formeln, herleitungen und erklärungen)

In diesem Versuchsteil wird eine Photodiode zum Messen des Lichtes verwendet. Diese hat die Eigenschaft sehr linear zur sein, so wie eine sehr hohe Geschwindigkeit. Jedoch ist der Photostrom sehr klein und die Messung ist stark von der Wellenlänge abhängig.

\subsubsection{Verwendete Geräte}
%(immer) eine skizze oder ein foto einfügen, die geräte/materialien !nummerieren! und z.b. eine legende dazu schreiben, besser wäre es das ganze in einem Fließtext gut zu beschreiben.
%falls am anfang des versuches nicht klar ist, was alles verwendet wird, wenn möglich erst am ende ein großes foto von den verwendeten materialien machen!\\

Es werden eine Photodiode und ein DMM verwendet.


\subsubsection{Versuchsaufbau}
%skizze zum versuchsaufbau (oder foto) einfügen,   es muss erklärt werden wie das ganze funktioniert und welche speziellen einstellungen verwendet wurden (z.b. welche knöpfe an den geräten für die messung verdreht wurden)

1 ist die Photodiode, 2 und 3 ist das DMM.

\begin{figure}[H] 
  \centering
    \includegraphics[ scale = 0.7]{Auf_1_3.PNG}
  	\caption[Schaltskizze zur Messung der Photodiode]{Schaltskizze zur Messung der Photodiode}
  \label{fig:auf_1_3}
\end{figure}

\subsubsection{Versuchsdurchführung}
%erklären, !was! wir machen, !warum! wir das machen und mit welchem ziel
%(wichtig) präzize erklären, wie bei dem versuch vorgegangen und was gemacht wurde

Die Photodiode wird in das DMM eingesetzt, einmal so das der Photostrom und ein mal die Photospannung gemessen werden. Dann wird wieder der Lichtstärkefilter von 20\% bis 100\%, mit 20\% Schritten verwendet und jeweils der Photostrom und die Photospannung gemessen.

\subsubsection{Messergebnisse}
%die messwerte in !übersichtlichen! tabellen angegeben
%zu viele kleine tabellen in große tabellen überführen!
%zu große tabellen mit dem [scale]-befehl scalieren oder (falls zu lang) in zwei kleinere tabellen aufteilen
%(wichtig) vor !jeder! tabelle sagen, was gemessen wurde und wie die fehler gewählt wurden und ausreichend !erklären!, !warum! wir unsere fehler grade so gewählt haben
\subsubsection{Auswertung}
%zuerst !alle! errechneten werte entweder in ganzen sätzen aufzählen, oder in tabellen (übersichtlicher) dargestellen, sowie auf die verwendeten formeln verweisen (die referenzierung der formel kann in der überschrift stehen)
%kurz erwähnen (vor der tabelle), warum wir das ganze ausrechnen bzw. was wir dort ausrechnen
%danach histogramme und plots erstellen, wobei wenn möglich funktionen durch die plots gelegt werden (zur not können auch splines benutzt werden, was aber angegeben werden muss)
%bei fits immer die funktion und das reduzierte chiquadrat mit angegeben, wobei auf verständlichkeit beim entziffern der zehnerpotenzen geachtet werden muss z.b. f(x)=(wert+-fehler)\cdot10^{irgendeine zahl}\cdot x + (wert+-fehler)\cdot10^{irgendeine zahl}
%bei jedem fit erklären, nach welchem zusammenhang gefittet wurde und warum!
%bei plots darauf achten, dass die achsenbeschriftung (auch die tics) die richtige größe haben und die legende im plot nicht die messwerte verdeckt
%kurz die aufgabenstellung abhandeln
%2-----------------------------------------------2
\subsubsection{Diskussion}
%(immer) die gemessenen werte und die bestimmten werte über die messfehler mit literaturwerten oder untereinander vergleichen
%in welchem fehlerintervall des messwertes liegt der literaturwert oder der vergleichswert?
%wie ist der relative anteil des fehlers am messwert und damit die qualität unserer messung?
%in einem satz erklären, wie gut unser fehler und damit unsere messung ist
%kurz erläutern, wie systematische fehler unsere messung beeinflusst haben könnten
%(wichtig) zum schluss ansprechen, in wie weit die ergebnisse mit der theoretischen vorhersage übereinstimmen
%--------------------------------------------------------------------------------------------
%falls tabellen mit den messwerten zu lang werden, kann die section mit den messwerten auch hinter der diskussion angefügt bzw. eine section mit dem anhang eingefügt werden.
%1-----------------------------------------------1




\subsection{Phototransistor}
%kurz das ziel dieses versuchsteiles ansprechen, damit keine zwei überschriften direkt übereinander stehen!
%bei schwierigeren versuchen kann auch der theoretische hintergrund erläutert werden. (mit formeln, herleitungen und erklärungen)

In diesem Versuchsteil wird die Lichtstärke mittels eines Phototransistors gemessen. Der Phototransistor hat den Vorteil, dass die Photostrom um das bis zu 100-fache verstärkt wird.

\subsubsection{Verwendete Geräte}
%(immer) eine skizze oder ein foto einfügen, die geräte/materialien !nummerieren! und z.b. eine legende dazu schreiben, besser wäre es das ganze in einem Fließtext gut zu beschreiben.
%falls am anfang des versuches nicht klar ist, was alles verwendet wird, wenn möglich erst am ende ein großes foto von den verwendeten materialien machen!\\

Es werden ein Widerstände, ein Phototransistor, ein DMM, eine Photodiode, ein Oszilloskop, ein Funktionsgenerator und eine Spannungsquelle verwendet.

\subsubsection{Versuchsaufbau}
%skizze zum versuchsaufbau (oder foto) einfügen,   es muss erklärt werden wie das ganze funktioniert und welche speziellen einstellungen verwendet wurden (z.b. welche knöpfe an den geräten für die messung verdreht wurden)

U0 ist eine 10V Spannungsquelle, R1 ein 1k$\Omega$ Widerstand, I ein DMM und T1 der Phototransistor.

\begin{figure}[H] 
  \centering
    \includegraphics[ scale = 0.7]{Auf_1_4_1.PNG}
  	\caption[Schaltskizze zur Messung mit dem Phototransistor]{Schaltskizze zur Messung mit dem Phototransistor}
  \label{fig:auf_1_4_1}
\end{figure}

1 ist der Funktionsgenerator, 2 ein 1k$\Omega$ Widerstand, 3 eine Photodiode, 4 ein 100k$\Omega$ Widerstand und 5 der Phototransistor.

\begin{figure}[H] 
  \centering
    \includegraphics[ scale = 0.7]{Auf_1_4_2.PNG}
  	\caption[Schaltskizze zur Messung mit  Photodiode und Phototransistor]{Schaltskizze zur Messung mit  Photodiode und Phototransistor}
  \label{fig:auf_1_4_2}
\end{figure}



\subsubsection{Versuchsdurchführung}
%erklären, !was! wir machen, !warum! wir das machen und mit welchem ziel
%(wichtig) präzize erklären, wie bei dem versuch vorgegangen und was gemacht wurde
\subsubsection{Messergebnisse}
%die messwerte in !übersichtlichen! tabellen angegeben
%zu viele kleine tabellen in große tabellen überführen!
%zu große tabellen mit dem [scale]-befehl scalieren oder (falls zu lang) in zwei kleinere tabellen aufteilen
%(wichtig) vor !jeder! tabelle sagen, was gemessen wurde und wie die fehler gewählt wurden und ausreichend !erklären!, !warum! wir unsere fehler grade so gewählt haben
\subsubsection{Auswertung}
%zuerst !alle! errechneten werte entweder in ganzen sätzen aufzählen, oder in tabellen (übersichtlicher) dargestellen, sowie auf die verwendeten formeln verweisen (die referenzierung der formel kann in der überschrift stehen)
%kurz erwähnen (vor der tabelle), warum wir das ganze ausrechnen bzw. was wir dort ausrechnen
%danach histogramme und plots erstellen, wobei wenn möglich funktionen durch die plots gelegt werden (zur not können auch splines benutzt werden, was aber angegeben werden muss)
%bei fits immer die funktion und das reduzierte chiquadrat mit angegeben, wobei auf verständlichkeit beim entziffern der zehnerpotenzen geachtet werden muss z.b. f(x)=(wert+-fehler)\cdot10^{irgendeine zahl}\cdot x + (wert+-fehler)\cdot10^{irgendeine zahl}
%bei jedem fit erklären, nach welchem zusammenhang gefittet wurde und warum!
%bei plots darauf achten, dass die achsenbeschriftung (auch die tics) die richtige größe haben und die legende im plot nicht die messwerte verdeckt
%kurz die aufgabenstellung abhandeln
%2-----------------------------------------------2
\subsubsection{Diskussion}
%(immer) die gemessenen werte und die bestimmten werte über die messfehler mit literaturwerten oder untereinander vergleichen
%in welchem fehlerintervall des messwertes liegt der literaturwert oder der vergleichswert?
%wie ist der relative anteil des fehlers am messwert und damit die qualität unserer messung?
%in einem satz erklären, wie gut unser fehler und damit unsere messung ist
%kurz erläutern, wie systematische fehler unsere messung beeinflusst haben könnten
%(wichtig) zum schluss ansprechen, in wie weit die ergebnisse mit der theoretischen vorhersage übereinstimmen
%--------------------------------------------------------------------------------------------
%falls tabellen mit den messwerten zu lang werden, kann die section mit den messwerten auch hinter der diskussion angefügt bzw. eine section mit dem anhang eingefügt werden.
%1-----------------------------------------------1




\subsection{Leuchtdiode als Lichtquelle}
%kurz das ziel dieses versuchsteiles ansprechen, damit keine zwei überschriften direkt übereinander stehen!
%bei schwierigeren versuchen kann auch der theoretische hintergrund erläutert werden. (mit formeln, herleitungen und erklärungen)
\subsubsection{Verwendete Geräte}
%(immer) eine skizze oder ein foto einfügen, die geräte/materialien !nummerieren! und z.b. eine legende dazu schreiben, besser wäre es das ganze in einem Fließtext gut zu beschreiben.
%falls am anfang des versuches nicht klar ist, was alles verwendet wird, wenn möglich erst am ende ein großes foto von den verwendeten materialien machen!\\
\subsubsection{Versuchsaufbau}
%skizze zum versuchsaufbau (oder foto) einfügen,   es muss erklärt werden wie das ganze funktioniert und welche speziellen einstellungen verwendet wurden (z.b. welche knöpfe an den geräten für die messung verdreht wurden)

R ist ein 1k$\Omega$ Widerstand, 1 ist die Leuchtdiode und U ein DMM.

\begin{figure}[H] 
  \centering
    \includegraphics[ scale = 0.7]{Auf_1_5_1.PNG}
  	\caption[Schaltskizze zur Messung mit  Photodiode und Phototransistor]{Schaltskizze zur Messung mit  Photodiode und Phototransistor}
  \label{fig:auf_1_5_1}
\end{figure}

1 ist die Photodiode, 2 der Phototransistor, R\_LED ist ein 470$\Omega$ Widerstand und R\_Ph-Tr ist ein 10 bis 100k$\Omega$ Widerstand.

\begin{figure}[H] 
  \centering
    \includegraphics[ scale = 0.7]{Auf_1_5_2.PNG}
  	\caption[Schaltskizze zur Messung mit  Photodiode und Phototransistor]{Schaltskizze zur Messung mit  Photodiode und Phototransistor}
  \label{fig:auf_1_5_2}
\end{figure}

\subsubsection{Versuchsdurchführung}
%erklären, !was! wir machen, !warum! wir das machen und mit welchem ziel
%(wichtig) präzize erklären, wie bei dem versuch vorgegangen und was gemacht wurde

\subsubsection{Messergebnisse}
%die messwerte in !übersichtlichen! tabellen angegeben
%zu viele kleine tabellen in große tabellen überführen!
%zu große tabellen mit dem [scale]-befehl scalieren oder (falls zu lang) in zwei kleinere tabellen aufteilen
%(wichtig) vor !jeder! tabelle sagen, was gemessen wurde und wie die fehler gewählt wurden und ausreichend !erklären!, !warum! wir unsere fehler grade so gewählt haben
\subsubsection{Auswertung}
%zuerst !alle! errechneten werte entweder in ganzen sätzen aufzählen, oder in tabellen (übersichtlicher) dargestellen, sowie auf die verwendeten formeln verweisen (die referenzierung der formel kann in der überschrift stehen)
%kurz erwähnen (vor der tabelle), warum wir das ganze ausrechnen bzw. was wir dort ausrechnen
%danach histogramme und plots erstellen, wobei wenn möglich funktionen durch die plots gelegt werden (zur not können auch splines benutzt werden, was aber angegeben werden muss)
%bei fits immer die funktion und das reduzierte chiquadrat mit angegeben, wobei auf verständlichkeit beim entziffern der zehnerpotenzen geachtet werden muss z.b. f(x)=(wert+-fehler)\cdot10^{irgendeine zahl}\cdot x + (wert+-fehler)\cdot10^{irgendeine zahl}
%bei jedem fit erklären, nach welchem zusammenhang gefittet wurde und warum!
%bei plots darauf achten, dass die achsenbeschriftung (auch die tics) die richtige größe haben und die legende im plot nicht die messwerte verdeckt
%kurz die aufgabenstellung abhandeln
%2-----------------------------------------------2
\subsubsection{Diskussion}
%(immer) die gemessenen werte und die bestimmten werte über die messfehler mit literaturwerten oder untereinander vergleichen
%in welchem fehlerintervall des messwertes liegt der literaturwert oder der vergleichswert?
%wie ist der relative anteil des fehlers am messwert und damit die qualität unserer messung?
%in einem satz erklären, wie gut unser fehler und damit unsere messung ist
%kurz erläutern, wie systematische fehler unsere messung beeinflusst haben könnten
%(wichtig) zum schluss ansprechen, in wie weit die ergebnisse mit der theoretischen vorhersage übereinstimmen
%--------------------------------------------------------------------------------------------
%falls tabellen mit den messwerten zu lang werden, kann die section mit den messwerten auch hinter der diskussion angefügt bzw. eine section mit dem anhang eingefügt werden.
%1-----------------------------------------------1





\section{Temperaturmessung}




\subsection{PTCs}
%kurz das ziel dieses versuchsteiles ansprechen, damit keine zwei überschriften direkt übereinander stehen!
%bei schwierigeren versuchen kann auch der theoretische hintergrund erläutert werden. (mit formeln, herleitungen und erklärungen)
\subsubsection{Verwendete Geräte}
%(immer) eine skizze oder ein foto einfügen, die geräte/materialien !nummerieren! und z.b. eine legende dazu schreiben, besser wäre es das ganze in einem Fließtext gut zu beschreiben.
%falls am anfang des versuches nicht klar ist, was alles verwendet wird, wenn möglich erst am ende ein großes foto von den verwendeten materialien machen!\\
\subsubsection{Verwendete Formeln}
%eine legende kann angefertigt werden, die selbstverständlichen buchstaben müssen nicht extra erklärt werden
%mit knappen erklärungen die !verwendeten! formeln, sowie die zugehörige fehlerrechnung einfügen
%2-----------------------------------------------2
%ab hier kann nochmal in einzelne versuchsteile unterteilt werden
\subsubsection{Versuchsaufbau}
%skizze zum versuchsaufbau (oder foto) einfügen,   es muss erklärt werden wie das ganze funktioniert und welche speziellen einstellungen verwendet wurden (z.b. welche knöpfe an den geräten für die messung verdreht wurden)
\subsubsection{Versuchsdurchführung}
%erklären, !was! wir machen, !warum! wir das machen und mit welchem ziel
%(wichtig) präzize erklären, wie bei dem versuch vorgegangen und was gemacht wurde

\subsubsection{Messergebnisse}
%die messwerte in !übersichtlichen! tabellen angegeben
%zu viele kleine tabellen in große tabellen überführen!
%zu große tabellen mit dem [scale]-befehl scalieren oder (falls zu lang) in zwei kleinere tabellen aufteilen
%(wichtig) vor !jeder! tabelle sagen, was gemessen wurde und wie die fehler gewählt wurden und ausreichend !erklären!, !warum! wir unsere fehler grade so gewählt haben
\subsubsection{Auswertung}
%zuerst !alle! errechneten werte entweder in ganzen sätzen aufzählen, oder in tabellen (übersichtlicher) dargestellen, sowie auf die verwendeten formeln verweisen (die referenzierung der formel kann in der überschrift stehen)
%kurz erwähnen (vor der tabelle), warum wir das ganze ausrechnen bzw. was wir dort ausrechnen
%danach histogramme und plots erstellen, wobei wenn möglich funktionen durch die plots gelegt werden (zur not können auch splines benutzt werden, was aber angegeben werden muss)
%bei fits immer die funktion und das reduzierte chiquadrat mit angegeben, wobei auf verständlichkeit beim entziffern der zehnerpotenzen geachtet werden muss z.b. f(x)=(wert+-fehler)\cdot10^{irgendeine zahl}\cdot x + (wert+-fehler)\cdot10^{irgendeine zahl}
%bei jedem fit erklären, nach welchem zusammenhang gefittet wurde und warum!
%bei plots darauf achten, dass die achsenbeschriftung (auch die tics) die richtige größe haben und die legende im plot nicht die messwerte verdeckt
%kurz die aufgabenstellung abhandeln
%2-----------------------------------------------2
\subsubsection{Diskussion}
%(immer) die gemessenen werte und die bestimmten werte über die messfehler mit literaturwerten oder untereinander vergleichen
%in welchem fehlerintervall des messwertes liegt der literaturwert oder der vergleichswert?
%wie ist der relative anteil des fehlers am messwert und damit die qualität unserer messung?
%in einem satz erklären, wie gut unser fehler und damit unsere messung ist
%kurz erläutern, wie systematische fehler unsere messung beeinflusst haben könnten
%(wichtig) zum schluss ansprechen, in wie weit die ergebnisse mit der theoretischen vorhersage übereinstimmen
%--------------------------------------------------------------------------------------------
%falls tabellen mit den messwerten zu lang werden, kann die section mit den messwerten auch hinter der diskussion angefügt bzw. eine section mit dem anhang eingefügt werden.
%1-----------------------------------------------1











\subsection{NTCs}
%kurz das ziel dieses versuchsteiles ansprechen, damit keine zwei überschriften direkt übereinander stehen!
%bei schwierigeren versuchen kann auch der theoretische hintergrund erläutert werden. (mit formeln, herleitungen und erklärungen)
\subsubsection{Verwendete Geräte}
%(immer) eine skizze oder ein foto einfügen, die geräte/materialien !nummerieren! und z.b. eine legende dazu schreiben, besser wäre es das ganze in einem Fließtext gut zu beschreiben.
%falls am anfang des versuches nicht klar ist, was alles verwendet wird, wenn möglich erst am ende ein großes foto von den verwendeten materialien machen!\\
\subsubsection{Verwendete Formeln}
%eine legende kann angefertigt werden, die selbstverständlichen buchstaben müssen nicht extra erklärt werden
%mit knappen erklärungen die !verwendeten! formeln, sowie die zugehörige fehlerrechnung einfügen
%2-----------------------------------------------2
%ab hier kann nochmal in einzelne versuchsteile unterteilt werden
\subsubsection{Versuchsaufbau}
%skizze zum versuchsaufbau (oder foto) einfügen,   es muss erklärt werden wie das ganze funktioniert und welche speziellen einstellungen verwendet wurden (z.b. welche knöpfe an den geräten für die messung verdreht wurden)
\subsubsection{Versuchsdurchführung}
%erklären, !was! wir machen, !warum! wir das machen und mit welchem ziel
%(wichtig) präzize erklären, wie bei dem versuch vorgegangen und was gemacht wurde

\subsubsection{Messergebnisse}
%die messwerte in !übersichtlichen! tabellen angegeben
%zu viele kleine tabellen in große tabellen überführen!
%zu große tabellen mit dem [scale]-befehl scalieren oder (falls zu lang) in zwei kleinere tabellen aufteilen
%(wichtig) vor !jeder! tabelle sagen, was gemessen wurde und wie die fehler gewählt wurden und ausreichend !erklären!, !warum! wir unsere fehler grade so gewählt haben
\subsubsection{Auswertung}
%zuerst !alle! errechneten werte entweder in ganzen sätzen aufzählen, oder in tabellen (übersichtlicher) dargestellen, sowie auf die verwendeten formeln verweisen (die referenzierung der formel kann in der überschrift stehen)
%kurz erwähnen (vor der tabelle), warum wir das ganze ausrechnen bzw. was wir dort ausrechnen
%danach histogramme und plots erstellen, wobei wenn möglich funktionen durch die plots gelegt werden (zur not können auch splines benutzt werden, was aber angegeben werden muss)
%bei fits immer die funktion und das reduzierte chiquadrat mit angegeben, wobei auf verständlichkeit beim entziffern der zehnerpotenzen geachtet werden muss z.b. f(x)=(wert+-fehler)\cdot10^{irgendeine zahl}\cdot x + (wert+-fehler)\cdot10^{irgendeine zahl}
%bei jedem fit erklären, nach welchem zusammenhang gefittet wurde und warum!
%bei plots darauf achten, dass die achsenbeschriftung (auch die tics) die richtige größe haben und die legende im plot nicht die messwerte verdeckt
%kurz die aufgabenstellung abhandeln
%2-----------------------------------------------2
\subsubsection{Diskussion}
%(immer) die gemessenen werte und die bestimmten werte über die messfehler mit literaturwerten oder untereinander vergleichen
%in welchem fehlerintervall des messwertes liegt der literaturwert oder der vergleichswert?
%wie ist der relative anteil des fehlers am messwert und damit die qualität unserer messung?
%in einem satz erklären, wie gut unser fehler und damit unsere messung ist
%kurz erläutern, wie systematische fehler unsere messung beeinflusst haben könnten
%(wichtig) zum schluss ansprechen, in wie weit die ergebnisse mit der theoretischen vorhersage übereinstimmen
%--------------------------------------------------------------------------------------------
%falls tabellen mit den messwerten zu lang werden, kann die section mit den messwerten auch hinter der diskussion angefügt bzw. eine section mit dem anhang eingefügt werden.
%1-----------------------------------------------1












\subsection{Thermoelemente}
%kurz das ziel dieses versuchsteiles ansprechen, damit keine zwei überschriften direkt übereinander stehen!
%bei schwierigeren versuchen kann auch der theoretische hintergrund erläutert werden. (mit formeln, herleitungen und erklärungen)
\subsubsection{Verwendete Geräte}
%(immer) eine skizze oder ein foto einfügen, die geräte/materialien !nummerieren! und z.b. eine legende dazu schreiben, besser wäre es das ganze in einem Fließtext gut zu beschreiben.
%falls am anfang des versuches nicht klar ist, was alles verwendet wird, wenn möglich erst am ende ein großes foto von den verwendeten materialien machen!\\
\subsubsection{Verwendete Formeln}
%eine legende kann angefertigt werden, die selbstverständlichen buchstaben müssen nicht extra erklärt werden
%mit knappen erklärungen die !verwendeten! formeln, sowie die zugehörige fehlerrechnung einfügen
%2-----------------------------------------------2
%ab hier kann nochmal in einzelne versuchsteile unterteilt werden
\subsubsection{Versuchsaufbau}
%skizze zum versuchsaufbau (oder foto) einfügen,   es muss erklärt werden wie das ganze funktioniert und welche speziellen einstellungen verwendet wurden (z.b. welche knöpfe an den geräten für die messung verdreht wurden)
\subsubsection{Versuchsdurchführung}
%erklären, !was! wir machen, !warum! wir das machen und mit welchem ziel
%(wichtig) präzize erklären, wie bei dem versuch vorgegangen und was gemacht wurde

\subsubsection{Messergebnisse}
%die messwerte in !übersichtlichen! tabellen angegeben
%zu viele kleine tabellen in große tabellen überführen!
%zu große tabellen mit dem [scale]-befehl scalieren oder (falls zu lang) in zwei kleinere tabellen aufteilen
%(wichtig) vor !jeder! tabelle sagen, was gemessen wurde und wie die fehler gewählt wurden und ausreichend !erklären!, !warum! wir unsere fehler grade so gewählt haben
\subsubsection{Auswertung}
%zuerst !alle! errechneten werte entweder in ganzen sätzen aufzählen, oder in tabellen (übersichtlicher) dargestellen, sowie auf die verwendeten formeln verweisen (die referenzierung der formel kann in der überschrift stehen)
%kurz erwähnen (vor der tabelle), warum wir das ganze ausrechnen bzw. was wir dort ausrechnen
%danach histogramme und plots erstellen, wobei wenn möglich funktionen durch die plots gelegt werden (zur not können auch splines benutzt werden, was aber angegeben werden muss)
%bei fits immer die funktion und das reduzierte chiquadrat mit angegeben, wobei auf verständlichkeit beim entziffern der zehnerpotenzen geachtet werden muss z.b. f(x)=(wert+-fehler)\cdot10^{irgendeine zahl}\cdot x + (wert+-fehler)\cdot10^{irgendeine zahl}
%bei jedem fit erklären, nach welchem zusammenhang gefittet wurde und warum!
%bei plots darauf achten, dass die achsenbeschriftung (auch die tics) die richtige größe haben und die legende im plot nicht die messwerte verdeckt
%kurz die aufgabenstellung abhandeln
%2-----------------------------------------------2
\subsubsection{Diskussion}
%(immer) die gemessenen werte und die bestimmten werte über die messfehler mit literaturwerten oder untereinander vergleichen
%in welchem fehlerintervall des messwertes liegt der literaturwert oder der vergleichswert?
%wie ist der relative anteil des fehlers am messwert und damit die qualität unserer messung?
%in einem satz erklären, wie gut unser fehler und damit unsere messung ist
%kurz erläutern, wie systematische fehler unsere messung beeinflusst haben könnten
%(wichtig) zum schluss ansprechen, in wie weit die ergebnisse mit der theoretischen vorhersage übereinstimmen
%--------------------------------------------------------------------------------------------
%falls tabellen mit den messwerten zu lang werden, kann die section mit den messwerten auch hinter der diskussion angefügt bzw. eine section mit dem anhang eingefügt werden.
%1-----------------------------------------------1











\section{Induktion}
%kurz das ziel dieses versuchsteiles ansprechen, damit keine zwei überschriften direkt übereinander stehen!
%bei schwierigeren versuchen kann auch der theoretische hintergrund erläutert werden. (mit formeln, herleitungen und erklärungen)
\subsubsection{Verwendete Geräte}
%(immer) eine skizze oder ein foto einfügen, die geräte/materialien !nummerieren! und z.b. eine legende dazu schreiben, besser wäre es das ganze in einem Fließtext gut zu beschreiben.
%falls am anfang des versuches nicht klar ist, was alles verwendet wird, wenn möglich erst am ende ein großes foto von den verwendeten materialien machen!\\
\subsubsection{Verwendete Formeln}
%eine legende kann angefertigt werden, die selbstverständlichen buchstaben müssen nicht extra erklärt werden
%mit knappen erklärungen die !verwendeten! formeln, sowie die zugehörige fehlerrechnung einfügen
%2-----------------------------------------------2
%ab hier kann nochmal in einzelne versuchsteile unterteilt werden
\subsubsection{Versuchsaufbau}
%skizze zum versuchsaufbau (oder foto) einfügen,   es muss erklärt werden wie das ganze funktioniert und welche speziellen einstellungen verwendet wurden (z.b. welche knöpfe an den geräten für die messung verdreht wurden)
\subsubsection{Versuchsdurchführung}
%erklären, !was! wir machen, !warum! wir das machen und mit welchem ziel
%(wichtig) präzize erklären, wie bei dem versuch vorgegangen und was gemacht wurde

\subsubsection{Messergebnisse}
%die messwerte in !übersichtlichen! tabellen angegeben
%zu viele kleine tabellen in große tabellen überführen!
%zu große tabellen mit dem [scale]-befehl scalieren oder (falls zu lang) in zwei kleinere tabellen aufteilen
%(wichtig) vor !jeder! tabelle sagen, was gemessen wurde und wie die fehler gewählt wurden und ausreichend !erklären!, !warum! wir unsere fehler grade so gewählt haben
\subsubsection{Auswertung}
%zuerst !alle! errechneten werte entweder in ganzen sätzen aufzählen, oder in tabellen (übersichtlicher) dargestellen, sowie auf die verwendeten formeln verweisen (die referenzierung der formel kann in der überschrift stehen)
%kurz erwähnen (vor der tabelle), warum wir das ganze ausrechnen bzw. was wir dort ausrechnen
%danach histogramme und plots erstellen, wobei wenn möglich funktionen durch die plots gelegt werden (zur not können auch splines benutzt werden, was aber angegeben werden muss)
%bei fits immer die funktion und das reduzierte chiquadrat mit angegeben, wobei auf verständlichkeit beim entziffern der zehnerpotenzen geachtet werden muss z.b. f(x)=(wert+-fehler)\cdot10^{irgendeine zahl}\cdot x + (wert+-fehler)\cdot10^{irgendeine zahl}
%bei jedem fit erklären, nach welchem zusammenhang gefittet wurde und warum!
%bei plots darauf achten, dass die achsenbeschriftung (auch die tics) die richtige größe haben und die legende im plot nicht die messwerte verdeckt
%kurz die aufgabenstellung abhandeln
%2-----------------------------------------------2
\subsubsection{Diskussion}
%(immer) die gemessenen werte und die bestimmten werte über die messfehler mit literaturwerten oder untereinander vergleichen
%in welchem fehlerintervall des messwertes liegt der literaturwert oder der vergleichswert?
%wie ist der relative anteil des fehlers am messwert und damit die qualität unserer messung?
%in einem satz erklären, wie gut unser fehler und damit unsere messung ist
%kurz erläutern, wie systematische fehler unsere messung beeinflusst haben könnten
%(wichtig) zum schluss ansprechen, in wie weit die ergebnisse mit der theoretischen vorhersage übereinstimmen
%--------------------------------------------------------------------------------------------
%falls tabellen mit den messwerten zu lang werden, kann die section mit den messwerten auch hinter der diskussion angefügt bzw. eine section mit dem anhang eingefügt werden.
%1-----------------------------------------------1






\section{Kraft}


\subsection{Elementare Dehnungsmessstreifen}
%kurz das ziel dieses versuchsteiles ansprechen, damit keine zwei überschriften direkt übereinander stehen!
%bei schwierigeren versuchen kann auch der theoretische hintergrund erläutert werden. (mit formeln, herleitungen und erklärungen)
\subsubsection{Verwendete Geräte}
%(immer) eine skizze oder ein foto einfügen, die geräte/materialien !nummerieren! und z.b. eine legende dazu schreiben, besser wäre es das ganze in einem Fließtext gut zu beschreiben.
%falls am anfang des versuches nicht klar ist, was alles verwendet wird, wenn möglich erst am ende ein großes foto von den verwendeten materialien machen!\\
\subsubsection{Verwendete Formeln}
%eine legende kann angefertigt werden, die selbstverständlichen buchstaben müssen nicht extra erklärt werden
%mit knappen erklärungen die !verwendeten! formeln, sowie die zugehörige fehlerrechnung einfügen
%2-----------------------------------------------2
%ab hier kann nochmal in einzelne versuchsteile unterteilt werden
\subsubsection{Versuchsaufbau}
%skizze zum versuchsaufbau (oder foto) einfügen,   es muss erklärt werden wie das ganze funktioniert und welche speziellen einstellungen verwendet wurden (z.b. welche knöpfe an den geräten für die messung verdreht wurden)
\subsubsection{Versuchsdurchführung}
%erklären, !was! wir machen, !warum! wir das machen und mit welchem ziel
%(wichtig) präzize erklären, wie bei dem versuch vorgegangen und was gemacht wurde

\subsubsection{Messergebnisse}
%die messwerte in !übersichtlichen! tabellen angegeben
%zu viele kleine tabellen in große tabellen überführen!
%zu große tabellen mit dem [scale]-befehl scalieren oder (falls zu lang) in zwei kleinere tabellen aufteilen
%(wichtig) vor !jeder! tabelle sagen, was gemessen wurde und wie die fehler gewählt wurden und ausreichend !erklären!, !warum! wir unsere fehler grade so gewählt haben
\subsubsection{Auswertung}
%zuerst !alle! errechneten werte entweder in ganzen sätzen aufzählen, oder in tabellen (übersichtlicher) dargestellen, sowie auf die verwendeten formeln verweisen (die referenzierung der formel kann in der überschrift stehen)
%kurz erwähnen (vor der tabelle), warum wir das ganze ausrechnen bzw. was wir dort ausrechnen
%danach histogramme und plots erstellen, wobei wenn möglich funktionen durch die plots gelegt werden (zur not können auch splines benutzt werden, was aber angegeben werden muss)
%bei fits immer die funktion und das reduzierte chiquadrat mit angegeben, wobei auf verständlichkeit beim entziffern der zehnerpotenzen geachtet werden muss z.b. f(x)=(wert+-fehler)\cdot10^{irgendeine zahl}\cdot x + (wert+-fehler)\cdot10^{irgendeine zahl}
%bei jedem fit erklären, nach welchem zusammenhang gefittet wurde und warum!
%bei plots darauf achten, dass die achsenbeschriftung (auch die tics) die richtige größe haben und die legende im plot nicht die messwerte verdeckt
%kurz die aufgabenstellung abhandeln
%2-----------------------------------------------2
\subsubsection{Diskussion}
%(immer) die gemessenen werte und die bestimmten werte über die messfehler mit literaturwerten oder untereinander vergleichen
%in welchem fehlerintervall des messwertes liegt der literaturwert oder der vergleichswert?
%wie ist der relative anteil des fehlers am messwert und damit die qualität unserer messung?
%in einem satz erklären, wie gut unser fehler und damit unsere messung ist
%kurz erläutern, wie systematische fehler unsere messung beeinflusst haben könnten
%(wichtig) zum schluss ansprechen, in wie weit die ergebnisse mit der theoretischen vorhersage übereinstimmen
%--------------------------------------------------------------------------------------------
%falls tabellen mit den messwerten zu lang werden, kann die section mit den messwerten auch hinter der diskussion angefügt bzw. eine section mit dem anhang eingefügt werden.
%1-----------------------------------------------1









\subsection{Dehnungsmessstreifen in Wheatstonscher Messbrücke, Wägezellen}
%kurz das ziel dieses versuchsteiles ansprechen, damit keine zwei überschriften direkt übereinander stehen!
%bei schwierigeren versuchen kann auch der theoretische hintergrund erläutert werden. (mit formeln, herleitungen und erklärungen)
\subsubsection{Verwendete Geräte}
%(immer) eine skizze oder ein foto einfügen, die geräte/materialien !nummerieren! und z.b. eine legende dazu schreiben, besser wäre es das ganze in einem Fließtext gut zu beschreiben.
%falls am anfang des versuches nicht klar ist, was alles verwendet wird, wenn möglich erst am ende ein großes foto von den verwendeten materialien machen!\\
\subsubsection{Verwendete Formeln}
%eine legende kann angefertigt werden, die selbstverständlichen buchstaben müssen nicht extra erklärt werden
%mit knappen erklärungen die !verwendeten! formeln, sowie die zugehörige fehlerrechnung einfügen
%2-----------------------------------------------2
%ab hier kann nochmal in einzelne versuchsteile unterteilt werden
\subsubsection{Versuchsaufbau}
%skizze zum versuchsaufbau (oder foto) einfügen,   es muss erklärt werden wie das ganze funktioniert und welche speziellen einstellungen verwendet wurden (z.b. welche knöpfe an den geräten für die messung verdreht wurden)
\subsubsection{Versuchsdurchführung}
%erklären, !was! wir machen, !warum! wir das machen und mit welchem ziel
%(wichtig) präzize erklären, wie bei dem versuch vorgegangen und was gemacht wurde

\subsubsection{Messergebnisse}
%die messwerte in !übersichtlichen! tabellen angegeben
%zu viele kleine tabellen in große tabellen überführen!
%zu große tabellen mit dem [scale]-befehl scalieren oder (falls zu lang) in zwei kleinere tabellen aufteilen
%(wichtig) vor !jeder! tabelle sagen, was gemessen wurde und wie die fehler gewählt wurden und ausreichend !erklären!, !warum! wir unsere fehler grade so gewählt haben
\subsubsection{Auswertung}
%zuerst !alle! errechneten werte entweder in ganzen sätzen aufzählen, oder in tabellen (übersichtlicher) dargestellen, sowie auf die verwendeten formeln verweisen (die referenzierung der formel kann in der überschrift stehen)
%kurz erwähnen (vor der tabelle), warum wir das ganze ausrechnen bzw. was wir dort ausrechnen
%danach histogramme und plots erstellen, wobei wenn möglich funktionen durch die plots gelegt werden (zur not können auch splines benutzt werden, was aber angegeben werden muss)
%bei fits immer die funktion und das reduzierte chiquadrat mit angegeben, wobei auf verständlichkeit beim entziffern der zehnerpotenzen geachtet werden muss z.b. f(x)=(wert+-fehler)\cdot10^{irgendeine zahl}\cdot x + (wert+-fehler)\cdot10^{irgendeine zahl}
%bei jedem fit erklären, nach welchem zusammenhang gefittet wurde und warum!
%bei plots darauf achten, dass die achsenbeschriftung (auch die tics) die richtige größe haben und die legende im plot nicht die messwerte verdeckt
%kurz die aufgabenstellung abhandeln
%2-----------------------------------------------2
\subsubsection{Diskussion}
%(immer) die gemessenen werte und die bestimmten werte über die messfehler mit literaturwerten oder untereinander vergleichen
%in welchem fehlerintervall des messwertes liegt der literaturwert oder der vergleichswert?
%wie ist der relative anteil des fehlers am messwert und damit die qualität unserer messung?
%in einem satz erklären, wie gut unser fehler und damit unsere messung ist
%kurz erläutern, wie systematische fehler unsere messung beeinflusst haben könnten
%(wichtig) zum schluss ansprechen, in wie weit die ergebnisse mit der theoretischen vorhersage übereinstimmen
%--------------------------------------------------------------------------------------------
%falls tabellen mit den messwerten zu lang werden, kann die section mit den messwerten auch hinter der diskussion angefügt bzw. eine section mit dem anhang eingefügt werden.
%1-----------------------------------------------1



\section{Luftdruckschwankungen/Schall}


\subsection{Dynamische Mikrofone}
%kurz das ziel dieses versuchsteiles ansprechen, damit keine zwei überschriften direkt übereinander stehen!
%bei schwierigeren versuchen kann auch der theoretische hintergrund erläutert werden. (mit formeln, herleitungen und erklärungen)
\subsubsection{Verwendete Geräte}
%(immer) eine skizze oder ein foto einfügen, die geräte/materialien !nummerieren! und z.b. eine legende dazu schreiben, besser wäre es das ganze in einem Fließtext gut zu beschreiben.
%falls am anfang des versuches nicht klar ist, was alles verwendet wird, wenn möglich erst am ende ein großes foto von den verwendeten materialien machen!\\
\subsubsection{Verwendete Formeln}
%eine legende kann angefertigt werden, die selbstverständlichen buchstaben müssen nicht extra erklärt werden
%mit knappen erklärungen die !verwendeten! formeln, sowie die zugehörige fehlerrechnung einfügen
%2-----------------------------------------------2
%ab hier kann nochmal in einzelne versuchsteile unterteilt werden
\subsubsection{Versuchsaufbau}
%skizze zum versuchsaufbau (oder foto) einfügen,   es muss erklärt werden wie das ganze funktioniert und welche speziellen einstellungen verwendet wurden (z.b. welche knöpfe an den geräten für die messung verdreht wurden)
\subsubsection{Versuchsdurchführung}
%erklären, !was! wir machen, !warum! wir das machen und mit welchem ziel
%(wichtig) präzize erklären, wie bei dem versuch vorgegangen und was gemacht wurde

\subsubsection{Messergebnisse}
%die messwerte in !übersichtlichen! tabellen angegeben
%zu viele kleine tabellen in große tabellen überführen!
%zu große tabellen mit dem [scale]-befehl scalieren oder (falls zu lang) in zwei kleinere tabellen aufteilen
%(wichtig) vor !jeder! tabelle sagen, was gemessen wurde und wie die fehler gewählt wurden und ausreichend !erklären!, !warum! wir unsere fehler grade so gewählt haben
\subsubsection{Auswertung}
%zuerst !alle! errechneten werte entweder in ganzen sätzen aufzählen, oder in tabellen (übersichtlicher) dargestellen, sowie auf die verwendeten formeln verweisen (die referenzierung der formel kann in der überschrift stehen)
%kurz erwähnen (vor der tabelle), warum wir das ganze ausrechnen bzw. was wir dort ausrechnen
%danach histogramme und plots erstellen, wobei wenn möglich funktionen durch die plots gelegt werden (zur not können auch splines benutzt werden, was aber angegeben werden muss)
%bei fits immer die funktion und das reduzierte chiquadrat mit angegeben, wobei auf verständlichkeit beim entziffern der zehnerpotenzen geachtet werden muss z.b. f(x)=(wert+-fehler)\cdot10^{irgendeine zahl}\cdot x + (wert+-fehler)\cdot10^{irgendeine zahl}
%bei jedem fit erklären, nach welchem zusammenhang gefittet wurde und warum!
%bei plots darauf achten, dass die achsenbeschriftung (auch die tics) die richtige größe haben und die legende im plot nicht die messwerte verdeckt
%kurz die aufgabenstellung abhandeln
%2-----------------------------------------------2
\subsubsection{Diskussion}
%(immer) die gemessenen werte und die bestimmten werte über die messfehler mit literaturwerten oder untereinander vergleichen
%in welchem fehlerintervall des messwertes liegt der literaturwert oder der vergleichswert?
%wie ist der relative anteil des fehlers am messwert und damit die qualität unserer messung?
%in einem satz erklären, wie gut unser fehler und damit unsere messung ist
%kurz erläutern, wie systematische fehler unsere messung beeinflusst haben könnten
%(wichtig) zum schluss ansprechen, in wie weit die ergebnisse mit der theoretischen vorhersage übereinstimmen
%--------------------------------------------------------------------------------------------
%falls tabellen mit den messwerten zu lang werden, kann die section mit den messwerten auch hinter der diskussion angefügt bzw. eine section mit dem anhang eingefügt werden.
%1-----------------------------------------------1



\subsection{Kondensator- oder Elektretmikrofone}
%kurz das ziel dieses versuchsteiles ansprechen, damit keine zwei überschriften direkt übereinander stehen!
%bei schwierigeren versuchen kann auch der theoretische hintergrund erläutert werden. (mit formeln, herleitungen und erklärungen)
\subsubsection{Verwendete Geräte}
%(immer) eine skizze oder ein foto einfügen, die geräte/materialien !nummerieren! und z.b. eine legende dazu schreiben, besser wäre es das ganze in einem Fließtext gut zu beschreiben.
%falls am anfang des versuches nicht klar ist, was alles verwendet wird, wenn möglich erst am ende ein großes foto von den verwendeten materialien machen!\\
\subsubsection{Verwendete Formeln}
%eine legende kann angefertigt werden, die selbstverständlichen buchstaben müssen nicht extra erklärt werden
%mit knappen erklärungen die !verwendeten! formeln, sowie die zugehörige fehlerrechnung einfügen
%2-----------------------------------------------2
%ab hier kann nochmal in einzelne versuchsteile unterteilt werden

Lückenfüller

\subsubsection{Versuchsaufbau}
%skizze zum versuchsaufbau (oder foto) einfügen,   es muss erklärt werden wie das ganze funktioniert und welche speziellen einstellungen verwendet wurden (z.b. welche knöpfe an den geräten für die messung verdreht wurden)

\begin{figure}[H] 
  \centering
    \includegraphics[ scale = 0.7]{Auf_5_2_1.PNG}
  	\caption[Schaltskizze für das Elektretmikrofon]{Schaltskizze für das Elektretmikrofon}
  \label{fig:auf_5_2_1}
\end{figure}

Lückenfüller

\begin{figure}[H] 
  \centering
    \includegraphics[ scale = 0.7]{Auf_5_2_2.PNG}
  	\caption[Schaltskizze für das Elektretmikrofon und das dynamische Mikrofon mit Operationsverstärker]{Schaltskizze für das Elektretmikrofon und das dynamische Mikrofon mit Operationsverstärker}
  \label{fig:auf_5_2_2}
\end{figure}

\subsubsection{Versuchsdurchführung}
%erklären, !was! wir machen, !warum! wir das machen und mit welchem ziel
%(wichtig) präzize erklären, wie bei dem versuch vorgegangen und was gemacht wurde

\subsubsection{Messergebnisse}
%die messwerte in !übersichtlichen! tabellen angegeben
%zu viele kleine tabellen in große tabellen überführen!
%zu große tabellen mit dem [scale]-befehl scalieren oder (falls zu lang) in zwei kleinere tabellen aufteilen
%(wichtig) vor !jeder! tabelle sagen, was gemessen wurde und wie die fehler gewählt wurden und ausreichend !erklären!, !warum! wir unsere fehler grade so gewählt haben
\subsubsection{Auswertung}
%zuerst !alle! errechneten werte entweder in ganzen sätzen aufzählen, oder in tabellen (übersichtlicher) dargestellen, sowie auf die verwendeten formeln verweisen (die referenzierung der formel kann in der überschrift stehen)
%kurz erwähnen (vor der tabelle), warum wir das ganze ausrechnen bzw. was wir dort ausrechnen
%danach histogramme und plots erstellen, wobei wenn möglich funktionen durch die plots gelegt werden (zur not können auch splines benutzt werden, was aber angegeben werden muss)
%bei fits immer die funktion und das reduzierte chiquadrat mit angegeben, wobei auf verständlichkeit beim entziffern der zehnerpotenzen geachtet werden muss z.b. f(x)=(wert+-fehler)\cdot10^{irgendeine zahl}\cdot x + (wert+-fehler)\cdot10^{irgendeine zahl}
%bei jedem fit erklären, nach welchem zusammenhang gefittet wurde und warum!
%bei plots darauf achten, dass die achsenbeschriftung (auch die tics) die richtige größe haben und die legende im plot nicht die messwerte verdeckt
%kurz die aufgabenstellung abhandeln
%2-----------------------------------------------2
\subsubsection{Diskussion}
%(immer) die gemessenen werte und die bestimmten werte über die messfehler mit literaturwerten oder untereinander vergleichen
%in welchem fehlerintervall des messwertes liegt der literaturwert oder der vergleichswert?
%wie ist der relative anteil des fehlers am messwert und damit die qualität unserer messung?
%in einem satz erklären, wie gut unser fehler und damit unsere messung ist
%kurz erläutern, wie systematische fehler unsere messung beeinflusst haben könnten
%(wichtig) zum schluss ansprechen, in wie weit die ergebnisse mit der theoretischen vorhersage übereinstimmen
%--------------------------------------------------------------------------------------------
%falls tabellen mit den messwerten zu lang werden, kann die section mit den messwerten auch hinter der diskussion angefügt bzw. eine section mit dem anhang eingefügt werden.
%1-----------------------------------------------1



\section{Feuchte}
%kurz das ziel dieses versuchsteiles ansprechen, damit keine zwei überschriften direkt übereinander stehen!
%bei schwierigeren versuchen kann auch der theoretische hintergrund erläutert werden. (mit formeln, herleitungen und erklärungen)
\subsubsection{Verwendete Geräte}
%(immer) eine skizze oder ein foto einfügen, die geräte/materialien !nummerieren! und z.b. eine legende dazu schreiben, besser wäre es das ganze in einem Fließtext gut zu beschreiben.
%falls am anfang des versuches nicht klar ist, was alles verwendet wird, wenn möglich erst am ende ein großes foto von den verwendeten materialien machen!\\
\subsubsection{Verwendete Formeln}
%eine legende kann angefertigt werden, die selbstverständlichen buchstaben müssen nicht extra erklärt werden
%mit knappen erklärungen die !verwendeten! formeln, sowie die zugehörige fehlerrechnung einfügen
%2-----------------------------------------------2
%ab hier kann nochmal in einzelne versuchsteile unterteilt werden
\subsubsection{Versuchsaufbau}
%skizze zum versuchsaufbau (oder foto) einfügen,   es muss erklärt werden wie das ganze funktioniert und welche speziellen einstellungen verwendet wurden (z.b. welche knöpfe an den geräten für die messung verdreht wurden)

\begin{figure}[H] 
  \centering
    \includegraphics[ scale = 0.7]{Auf_6_1.PNG}
  	\caption[Schaltskizze für die Feuchtigkeitsmessung]{Schaltskizze für die Feuchtigkeitsmessung}
  \label{fig:auf_6}
\end{figure}

\subsubsection{Versuchsdurchführung}
%erklären, !was! wir machen, !warum! wir das machen und mit welchem ziel
%(wichtig) präzize erklären, wie bei dem versuch vorgegangen und was gemacht wurde

\subsubsection{Messergebnisse}
%die messwerte in !übersichtlichen! tabellen angegeben
%zu viele kleine tabellen in große tabellen überführen!
%zu große tabellen mit dem [scale]-befehl scalieren oder (falls zu lang) in zwei kleinere tabellen aufteilen
%(wichtig) vor !jeder! tabelle sagen, was gemessen wurde und wie die fehler gewählt wurden und ausreichend !erklären!, !warum! wir unsere fehler grade so gewählt haben
\subsubsection{Auswertung}
%zuerst !alle! errechneten werte entweder in ganzen sätzen aufzählen, oder in tabellen (übersichtlicher) dargestellen, sowie auf die verwendeten formeln verweisen (die referenzierung der formel kann in der überschrift stehen)
%kurz erwähnen (vor der tabelle), warum wir das ganze ausrechnen bzw. was wir dort ausrechnen
%danach histogramme und plots erstellen, wobei wenn möglich funktionen durch die plots gelegt werden (zur not können auch splines benutzt werden, was aber angegeben werden muss)
%bei fits immer die funktion und das reduzierte chiquadrat mit angegeben, wobei auf verständlichkeit beim entziffern der zehnerpotenzen geachtet werden muss z.b. f(x)=(wert+-fehler)\cdot10^{irgendeine zahl}\cdot x + (wert+-fehler)\cdot10^{irgendeine zahl}
%bei jedem fit erklären, nach welchem zusammenhang gefittet wurde und warum!
%bei plots darauf achten, dass die achsenbeschriftung (auch die tics) die richtige größe haben und die legende im plot nicht die messwerte verdeckt
%kurz die aufgabenstellung abhandeln
%2-----------------------------------------------2
\subsubsection{Diskussion}
%(immer) die gemessenen werte und die bestimmten werte über die messfehler mit literaturwerten oder untereinander vergleichen
%in welchem fehlerintervall des messwertes liegt der literaturwert oder der vergleichswert?
%wie ist der relative anteil des fehlers am messwert und damit die qualität unserer messung?
%in einem satz erklären, wie gut unser fehler und damit unsere messung ist
%kurz erläutern, wie systematische fehler unsere messung beeinflusst haben könnten
%(wichtig) zum schluss ansprechen, in wie weit die ergebnisse mit der theoretischen vorhersage übereinstimmen
%--------------------------------------------------------------------------------------------
%falls tabellen mit den messwerten zu lang werden, kann die section mit den messwerten auch hinter der diskussion angefügt bzw. eine section mit dem anhang eingefügt werden.
%1-----------------------------------------------1





\section{Fazit}
%im fazit nochmal alles zusammenfassen und den verlauf der messung abschätzen
%gravierende sytematische probleme bei den messungen nochmal betonen und die wertigkeit unserer ergebnisse einordnen
\end{document}

